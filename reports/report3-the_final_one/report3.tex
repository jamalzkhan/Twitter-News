\documentclass[a4paper,12pt]{article}
\usepackage{fullpage}
\usepackage{hyperref}
\usepackage{url}
\usepackage{graphicx}
%\usepackage{polski}
\usepackage[utf8]{inputenc}

\setlength{\parindent}{0pt}
\addtolength{\parskip}{\baselineskip}

\title{3rd Year Group Project\\Report Three\\}

\author{
    \small{Rafał Szymański}\\
  	\and
    \small{Maciek Albin}\\
    \and
    \small{Sam Wong}\\
    \and  
    \small{Suhaib Sarmad}\\
		\and
		\small{Jamal Khan}\\
		\and
		\small{\{rs2909, mja108, sw2309, sss308, jzk09\}@doc.ic.ac.uk}
		\and
		\\Department of Computing - Imperial College London
}

\date{}

\begin{document} 
	\maketitle
	
	\section{Testing}
	
	  \subsection{Logging and Debugging}
	
	  Instead of using standard print statements to output the state of our system, which is composed of the RSS, Analysis and Twitter Thread, we incorporated a flexible event logging system.\\
	We are able to leave our system and if an error occurs we can trace exactly which thread caused it and at which point in the program as opposed to the usual debugging.
  The logger was also incorporated so that we have different levels of importance in logging i.e. have logging for general information of what is going on in the system and logging for errors. We are also able to log messages to different output sinks including the console and files.
	
	  \begin{figure}[ht!]
				  \centering
					  \includegraphics[scale=0.4]{logger.png}
				    \caption{Customised Logger}
	  \end{figure}
	
	  \subsection{Unit Testing}
	
	For testing various parts of the system we used unit tests to check that the code we were writing did not brake existing functionality that we already had.
	
	\section{General Validation}
	
	\section{Managerial Documentation}
	
		\subsection{Collaboration Tools Used}
		
			\subsubsection{Git}
			
			We have used the git version control system for keeping track of the project, for easily reverting if there is a problem, and for having a very quick deploy mechanism. Whenever a code push occurs to our git repository, we have a git post-receive hook that copies the static files appropriately, and restarts the appropriate processes. This means that as soon as a code push occurs, the new version is live.
			
			\subsubsection{Trello}
			
			Trello\footnote{\url{http://trello.com}} is a very good piece of software by FogCreek. It is a digital board that allows you to create post-its and write the product backlog, and move tasks between the product backlog, the current iteration, and the finished tasks. We plan to extract and append the Trello history to our final report, to show progress. Here is a screenshot of how Trello works:
			
			\begin{figure}[ht!]
						\centering
							\includegraphics[scale=0.4]{trello1.png}
						\caption{Trello project management board}
			\end{figure}

		
		\subsection{Management policies}
		
		\subsection{Management of knowledge transfer within the group}
	
	\subsection{Member Contributions}
	  \subsubsection{Maciek Albin}
	    \begin{tabular}{l | p{10cm} r}
	     \emph{\large Date} & \emph{\large Comments} & \emph{\large Hours}\\
	     \hline
	     12/10/2011 & Advanced network generation for follower/folowee relations. & 3\\
	     20/10/2011 & More work on the network generation. & 4\\
	     28/10/2011 & Initial internal API definition and basic implementation. & 2\\
	     28/10/2011 & Fixes to inter thread communication. & 2\\
	     02/11/2011 & Refactoring. & 3\\
	     02/11/2011 & Implemented keyword generation using Alchemy API. & 2\\
	     03/11/2011 & Implemented basic analysis thread. & 4\\
	     04/11/2011 & Assigning tweets to stories in analysis thread. & 3\\
	     04/11/2011 & Added short summaries and links to stories DB. & 2\\
	     08/11/2011 & Added wordclouds to the API. & 1\\
	     10/11/2011 & Refactoring. & 3\\
	     30/11/2011 & Fixed bugs in the logger. & 2
	    \end{tabular}
	    
	  \subsubsection{Jamal Khan}
	  \begin{tabular}{l | p{10cm} r}
     \emph{\large Date} & \emph{\large Comments} & \emph{\large Hours}\\
     \hline
	  11/10/2011 & Initial creation of the RSS Fetcher using python module feedparser. & 2\\
    31/10/2011 & Keyword extraction completed for each story using Alchemy API. & 4\\
    4/11/2011 & Implemented Flask Server creating API for the front end for the site as well as relevant database calls to fetch stories. & 7\\
    10/11/2011 & Added functionality of getting stories by timestamp for the API. & 2\\
    25/11/2011 & Completed flexible logger for each thread, so we can see exactly what is going on. & 7\\
    12/11/2011 & Unit testing for RSS Fetcher completed. & 4
  \end{tabular}
	
	  \subsubsection{Rafał Szymański}
	    \begin{tabular}{l | p{10cm} r}
	     \emph{\large Date} & \emph{\large Comments} & \emph{\large Hours}\\
	     \hline
	     10/10/2011 & Initial commit for the project. Basic implementation in Python of a script that given a keyword  fetches the live streaming API for the given keyword and writes it to the Mongo database. & 6\\
       11/10/2011 & Bug fixes and a short readme. & 2\\
       12/10/2011 & Fetching the follower/folowee graph and saving to a file. Basic analysis with NetworkX. & 3\\
       20/10/2011 & More working on generating the user graph. & 2\\
       21/10/2011 & Big update. Gets headlines from Google news RSS, generates keywords using Termtopia, sets up twitter stream, saves to Mongo and repeats every 5 minutes. & 7\\
       01/11/2011 & Improving the threading of the program - added condition variables and fixed previous threading bugs. & 4\\
       04/11/2011 & Working on instant deployment environment using Flask, Nginx, uwsgi, and a couple other technologies. Took forever to do but afterwards, after every git push, the new version is live straight away. & 10\\
       07/11/2011 & Improvements to the instant deployment - new git hook, and a control script that allows remote restarting and clearing of the database. & 4\\
       08/11/2011 & Wordcloud generation. Goes through each tweet, finds the most occurring words and adds that data for each analysis period. & 4\\
       11/11/2011 & Improvements to control script. & 2\\
       30/11/2011 & Sentiment analysis for each period. Gets sentiment using an API and adds to Mongo. & 5\\
       04/12/2011 & Top tweets based on the number of retweets. & 2\\
       04/12/2011 & Added all my new data to the frontend API. & 2
	    \end{tabular}
	  
	  \subsubsection{Sam Wong}
	  \begin{tabular}{l | p{10cm} r}
     \emph{\large Date} & \emph{\large Comments} & \emph{\large Hours}\\
     \hline
	   12/10/2011 & Python development & 1\\
     02/11/2011 & UI Prototype 1 & 3\\
     09/11/2011 & UI Prototype 2 & 4\\
     20/11/2011 & UI Prototype 3 & 5\\
     21/11/2011 & UI debugging & 1\\
     25/11/2011 & UI features experimentation & 1\\
     01/12/2011 & UI features refinement & 1\\
     01/12/2011 & Auto-refresh & 2\\
     01/12/2011 & infinite scroll proof of concept & 3\\
    \end{tabular}
  

\end{document}
