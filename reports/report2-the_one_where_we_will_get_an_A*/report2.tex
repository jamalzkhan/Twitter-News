\documentclass[a4paper,12pt]{article}
\usepackage{fullpage}
\usepackage{hyperref}
\usepackage{url}
\usepackage{graphicx}
\usepackage{polski}
\usepackage[utf8]{inputenc}

\setlength{\parindent}{0pt}
\addtolength{\parskip}{\baselineskip}

\title{3rd Year Group Project\\Report Two\\\emph{Progress \& Revisions}}

\author{
    \small{Rafał Szymański}\\
  	\and
    \small{Maciek Albin}\\
    \and
    \small{Sam Wong}\\
    \and
    \small{Suhaib Sarmad}\\
		\and
		\small{Jamal Khan}\\
		\and
		\small{\{rs2909, mja108, sw2309, sss308, jzk09\}@doc.ic.ac.uk}
		\and
		\\Department of Computing - Imperial College London
}

\date{}

\begin{document} 
	\maketitle
	
	\section{Report I Amendment}
	
	\section{Progress}
	
	\section{Revisions}
	
	\section{People Management}
	
	As it often happens with group projects, problems relating group interactions and dynamics surface. Firstly, in our group, everyone knows each other well and is good friends with everyone else, which reduces the authoritative attitude of any one group member towards others. In a real life project, you have a manager with an authoritative position to do things such as lay you off, which certainly induces at least a little bit of motivation. In a university project where all the members are good friends, it is harder to establish an authority that is able to motivate others, especially during a time where everyone has a lot of other personal tasks at hand.
	
	During this term all of us have lot of other personal responsibilities, including many courseworks for other subjects, setting up and studying for internship interviews for next year, and teaching PMT classes, among many others. Considering every one of us has so many personal tasks that directly affect just him and not the group, we are seeing decreased motivation towards the project from all group members. We are directly experiencing \emph{Social Loafing}\footnote{\url{http://en.wikipedia.org/wiki/Social_loafing}}, which is "the phenomenon of people exerting less effort to achieve a goal when they work in a group than when they work alone." It still seems that we are having problems keeping to the iteration task assignment on Trello\footnote{\url{http://trello.com}}, ie we are not completing the iteration subtasks within the timeline. The root cause of this is mostly likely related to the above explained \emph{social loathing}, and diffusion of responsibility within a group without a clear authoritative figure.
	
	Our current proposed solution is rationalising to everyone the relative importance of this project compared to all other coursework - $440/1700$ points. Receiving a \textbf{C} for our first report was disappointing and a blow to our motivation, as none of us had gotten a C in Imperial before, but on the other hand it was a sign we need increase the quality of our work and approach the project more seriously.
	
	We have separated our group of five people as follows: 2 for UI, and 3 for backend, so everyone is doing what they are more comfortable with. Both of these subteams do pair programming. Many times one of us knows something useful, for example a specific MongoDB query syntax, and can help the other without having to resolve to Google. This ensures a good flow of information between the team, and levels out our knowledge. Nevertheless, what we still thrive to have is a \emph{bus count}\footnote{\url{bit.ly/j1zquA} - quite interesting article.}\footnote{Bus count - "how many people in your team have to get hit by a bus before you’re all dead in the water"} of the size of our group - we want everyone to understand, and if needed, to work on any part of our stack.
	
	We try to establish fair contribution by everyone by assigning to everyone tasks on Trello. We have planned to have frequent scrum meetings regarding the completion of this tasks, but instead, since we all see each other during lectures anyways, we usually discuss the progress without having a dedicated meeting. What we need to do is actually get into the habit of having a dedicated progress meeting more frequently, where everyone has to showcase what they achieved in the previous 2 or 3 days.
	
	\section{Ethical and Environmental Impact}	

\end{document}