\documentclass{report}
\usepackage{fullpage}
\usepackage{amssymb}
\usepackage[utf8]{inputenc}
\usepackage{url}

\setlength{\parindent}{0pt}
\addtolength{\parskip}{\baselineskip}

\title{Twitter News Generation - Final Report}

\author{
    \small{Rafał Szymański}\\
  	\and
    \small{Maciek Albin}\\
    \and
    \small{Sam Wong}\\
    \and  
    \small{Suhaib Sarmad}\\
		\and
		\small{Jamal Khan}\\
		\and
		\small{\{rs2909, mja108, sw2309, sss308, jzk09\}@doc.ic.ac.uk}
		\and
		\\Department of Computing - Imperial College London
}



\begin{document} 
	\maketitle
	\tableofcontents
	\newpage

	\section{High Level Overview}
	
	\section{Technical Overview}
	
	\section{Software Engineering Issues}
	
		\subsection{Summary of each members contribution}
		
		All the members fairly contributed to the project. As stated beforehand, we had a clear distinction between the people working on the frontend and backend.
		
		  \subsubsection{Maciek Albin}
		    \begin{tabular}{l | p{10cm} r}
		     \emph{\large Date} & \emph{\large Comments} & \emph{\large Hours}\\
		     \hline
		     12/10/2011 & Advanced network generation for follower/folowee relations. & 3\\
		     20/10/2011 & More work on the network generation. & 4\\
		     28/10/2011 & Initial internal API definition and basic implementation. & 2\\
		     28/10/2011 & Fixes to inter thread communication. & 2\\
		     02/11/2011 & Refactoring. & 3\\
		     02/11/2011 & Implemented keyword generation using Alchemy API. & 2\\
		     03/11/2011 & Implemented basic analysis thread. & 4\\
		     04/11/2011 & Assigning tweets to stories in analysis thread. & 3\\
		     04/11/2011 & Added short summaries and links to stories DB. & 2\\
		     08/11/2011 & Added wordclouds to the API. & 1\\
		     10/11/2011 & Refactoring. & 3\\
		     30/11/2011 & Fixed bugs in the logger. & 2
		    \end{tabular}

		  \subsubsection{Jamal Khan}
		  \begin{tabular}{l | p{10cm} r}
	     \emph{\large Date} & \emph{\large Comments} & \emph{\large Hours}\\
	     \hline
		  11/10/2011 & Initial creation of the RSS Fetcher using python module feedparser. & 2\\
	    31/10/2011 & Keyword extraction completed for each story using Alchemy API. & 4\\
	    4/11/2011 & Implemented Flask Server creating API for the front end for the site as well as relevant database calls to fetch stories. & 7\\
	    10/11/2011 & Added functionality of getting stories by timestamp for the API. & 2\\
	    25/11/2011 & Completed flexible logger for each thread, so we can see exactly what is going on. & 7\\
	    12/11/2011 & Unit testing for RSS Fetcher completed. & 4
	  \end{tabular}

	  \subsubsection{Suhaib Sarmad}
	    \begin{tabular}{l | p{10cm} r}
	     \emph{\large Date} & \emph{\large Comments} & \emph{\large Hours}\\
	     \hline
	     27/10/2011 & Resource research and UI mockup & 4\\
	     28/10/2011 & Basic grid UI proof of concept & 3\\
	     09/11/2011 & Implementation of basic jQuery Masonry (tiled) UI & 2\\
	     10/11/2011 & Create tiles in UI from server news API & 4\\
	     11/11/2011 & CSS/HTML Tile layout, split into title/summary, picture, sentiment, word cloud, tweet list & 5\\
	     11/11/2011 & UI debugging and cross-browser compatibility: works on all tested browsers except IE & 2\\
	     12/11/2011 & Added tweets to tweet list using Twitter REST API and keywords from server API, added custom jQuery scrollbars & 1\\
	     12/11/2011 & Added pictures to tiles using Google Images API and article titles & 2\\
	     12/11/2011 & Generating html5 word cloud from keywords from server API & 3\\
	     19/11/2011 & CSS3 transitions research and testing for big picture UI & 2\\
	     01/12/2011 & Automatic refreshing and fetching of new news articles from server & 4\\
	     09/12/2011 & Sentiment bar prototype & 1
	    \end{tabular}


		  \subsubsection{Rafał Szymański}
		    \begin{tabular}{l | p{10cm} r}
		     \emph{\large Date} & \emph{\large Comments} & \emph{\large Hours}\\
		     \hline
		     10/10/2011 & Initial commit for the project. Basic implementation in Python of a script that given a keyword  fetches the live streaming API for the given keyword and writes it to the Mongo database. & 6\\
	       11/10/2011 & Bug fixes and a short readme. & 2\\
	       12/10/2011 & Fetching the follower/folowee graph and saving to a file. Basic analysis with NetworkX. & 3\\
	       20/10/2011 & More working on generating the user graph. & 2\\
	       21/10/2011 & Big update. Gets headlines from Google news RSS, generates keywords using Termtopia, sets up twitter stream, saves to Mongo and repeats every 5 minutes. & 7\\
	       01/11/2011 & Improving the threading of the program - added condition variables and fixed previous threading bugs. & 4\\
	       04/11/2011 & Working on instant deployment environment using Flask, Nginx, uwsgi, and a couple other technologies. Took forever to do but afterwards, after every git push, the new version is live straight away. & 10\\
	       07/11/2011 & Improvements to the instant deployment - new git hook, and a control script that allows remote restarting and clearing of the database. & 4\\
	       08/11/2011 & Wordcloud generation. Goes through each tweet, finds the most occurring words and adds that data for each analysis period. & 4\\
	       11/11/2011 & Improvements to control script. & 2\\
	       30/11/2011 & Sentiment analysis for each period. Gets sentiment using an API and adds to Mongo. & 5\\
	       04/12/2011 & Top tweets based on the number of retweets. & 2\\
	       04/12/2011 & Added all my new data to the frontend API. & 2
		    \end{tabular}

		  \subsubsection{Sam Wong}
		  \begin{tabular}{l | p{10cm} r}
	     \emph{\large Date} & \emph{\large Comments} & \emph{\large Hours}\\
	     \hline
		   12/10/2011 & Python development & 1\\
	     02/11/2011 & UI Prototype 1 & 3\\
	     09/11/2011 & UI Prototype 2 & 4\\
	     20/11/2011 & UI Prototype 3 & 5\\
	     21/11/2011 & UI debugging & 1\\
	     25/11/2011 & UI features experimentation & 1\\
	     01/12/2011 & UI features refinement & 1\\
	     01/12/2011 & Auto-refresh & 2\\
	     01/12/2011 & infinite scroll proof of concept & 3\\
	    \end{tabular}
	
	\section{Validation and Conclusions}
	
	\section{Bibliography and Tools used}
	
	Taking into account the fact that this wasn't a research project requiring extensive external material, we do not provide a full biography, but provide links to sources we have used and consulted when working on this software engineering project.
	
	\begin{itemize}
		\item Backend Resources
		\begin{itemize}
			\item 	\url{http://pyunit.sourceforge.net/} - Unit testing
			\item 	\url{http://nginx.org/} - Backend server
			\item 	\url{http://projects.unbit.it/uwsgi/} - For nginx to speak with python
			\item 	\url{http://alchemyapi.com} - API for keyword extraction
			\item 	\url{http://mongodb.com} - Data store used 
			\item 	\url{http://www.logilab.org/857} - Pylint
		\end{itemize}
		
		
		\item Frontend Resources
		\begin{itemize}
			\item 	\url{http://validator.w3.org/} - HTML Validator
			\item 	\url{http://jquery.com/} - jQuery
			\item 	\url{http://masonry.desandro.com/} - jQuery Masonry Plugin
		\end{itemize}
		
		\item Other Resources
		\begin{itemize}
			\item 	\url{http://trello.com} - Online task board
			\item 	\url{http://www.python.org/doc/} - Python documentation
			\item 	\url{http://git-scm.com/} - Version control system used
			\item 	\url{https://github.com/} - For storing code + wiki
			\item 	\url{http://flask.pocoo.org/} - Python web framework
			\item 	\url{http://news.google.com/} - News source
			\item 	Programming Collective Intelligence, Toby Segaran, OReilly Media, 2007
			\item 	Mining the Social Web, Matthew A. Russell, OReilly Media, 2011
		\end{itemize}
		
	\end{itemize} 
	
	\section{Appendix}
	
	\subsection{Team Meetings}
	\begin{tabular}{c | l p{7cm} r}
    \emph{\large Date} &  \emph{\large Venue} &  \emph{\large Subject} &  \emph{\large Attended}\\
    \hline
    14/10/2011 & Lab Round Table & Choose which project to do & \(G\)\\
    18/10/2011 & Skypeland & Choose which project to do & \(G\)\\
    21/10/2011 & Lab Round Table & Discovered the original plan is infeasible due to constraints set by Twitter, draft backup plan & \(G\)\\
    27/10/2011 & Lab Round Table & Plan approved, Design Architecture & \(G \smallsetminus \{\texttt{Suhaib}\}\)\\
    3/11/2011 & Lab Round Table & Progress Report (Backend, and UI Prototype I) & \(G \smallsetminus \{\texttt{Rafal}\}\)\\
    10/11/2011 & Lab Round Table & UI Prototype II presentation, API requests and design & \(G \smallsetminus \{\texttt{Maciej}\}\)\\
    17/11/2011 & Lab Round Table & UI Prototype III. Post-JB-meeting discussion & \(G \smallsetminus \{\texttt{Jamal}\}\)\\
    24/11/2011 & Skypeland & Incremental improvements & \(G \smallsetminus \{\texttt{Sam}\}\)\\
    1/12/2011 & Skypeland & Extensions ideas & \(G \smallsetminus \{\texttt{Suhaib}\}\)\\
    8/12/2011 & Skypeland & Features prioritisation & \(G\)\\
  \end{tabular}

  \(G=\{\texttt{Rafal}, \texttt{Sam}, \texttt{Suhaib}, \texttt{Jamal}, \texttt{Maciek}\}\)\\


	

\end{document}
